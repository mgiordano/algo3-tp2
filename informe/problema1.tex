\section{Problema 1}


\subsection{Primeros Acercamientos al Problema}


\subsection{Análisis del Algoritmo}

\indent La idea detrás de nuestra solución...

\textbf{Pseudocódigo}:

\begin{algorithm}
\caption{buscarPeso (\textbf{in/out} mapa: \textsl{Mapa}) $\rightarrow$ res: \textsl{int}}
\begin{algorithmic}[1]

\STATE $heap \leftarrow$ crear heap de tramos vacío
\STATE $estado$ $origen \leftarrow confirmado$
\STATE agregar tramos salientes de $origen$ al $heap$

\WHILE{$estado$ $destino = no$ $confirmado$}
	\STATE $tramo \leftarrow$ pop elemento del $heap$
	\STATE $ciudad\_hasta \leftarrow$ ciudad a la que llega $tramo$
	\STATE $ciudad\_desde \leftarrow$ ciudad desde la que sale el $tramo$
	\IF{$estado$ $ciudad\_hasta = confirmado$} \STATE continuar
	\ENDIF
	\STATE $estado$ $ciudad\_hasta \leftarrow confirmado$
	\STATE $peso$ $ciudad\_hasta \leftarrow min(peso$ $tramo,peso$ $ciudad\_desde)$
	\STATE agregar tramos salientes de $ciudad$ al $heap$
	
\ENDWHILE
\RETURN $peso$ $destino$  
\end{algorithmic}
\end{algorithm}

\subsection{Demostración de Correctitud}


\subsection{Complejidad}

Antes de comenzar el análisis de la complejidad del algoritmo principal \textsl{buscarPeso}, es necesario conocer el costo de la construcción y manejo de las estructuras involucradas.\\

\textbf{Tramo.} Esta estructura representa los caminos que unen a las distintas ciudades del problema. Poseen un campo \textsl{peso} que indica, con un entero, cuál es la cantidad máxima que se puede trasladar por esa ruta, mientras que en los campos \textsl{desde} y \textsl{hasta} almacena las respectivas ciudades que se encuentran en los extremos de ese tramo. Vale notar que a pesar de la semántica del problema en que un camino de A hasta B también es válido desde B hasta A, nuestro modelado de la estructura define un tramo dirigido de un extremo 'desde' hacia otro 'hasta' y no cuenta la vuelta. Es tarea de otra estructura crear, para un tramo dado, dos tramos independientes que noten la ida y la vuelta.\\
\indent El constructor toma como parámetros las ciudades 'desde' y 'hasta', así como también el entero referente al peso, y asigna cada objeto al respectivo campo, lo cual simplemente cuesta tiempo constante, o sea O($1$). Dado que el algoritmo principal del problema necesita almacenar algunos tramos en un heap, existe también una función comparadora que determina el orden entre los mismos. La función toma dos objetos de la estructura y devuelve $1$ si la primera admite menos peso, $-$1 si admite más peso, o $0$ si ambas admiten el mismo peso. Dado que se trata de tres comparaciones de enteros, la complejidad de esta operación también es O($1$).\\
\indent \textbf{Ciudad.} Esta clase permite modelar las ciudades del mapa asignándole los siguientes atributos: un string \textsl{nombre}; un booleano \textsl{estado} que permite determinar si ya se pasó por esa ciudad; un entero \textsl{peso} que, si el estado de la ciudad es 'confirmado', representa el máximo peso que se puede llevar hasta ella desde la ciudad 'origen', y una lista de tramos que contiene las aristas que conectan esa ciudad con sus vecinos. \\
\indent Su constructor toma únicamente un string nombre. Asigna este valor como identificador del objeto en su respectivo campo, luego setea el estado como falso (es decir 'no confirmado') y además inicializa el peso en  $\infty$. Luego crea una lista vacía de tramos y la asigna al campo \textsl{vecinos}. Dado que la lista de tramos está está representada por la interfaz de Java List$<T>$, que a su vez está implementada sobre la clase ArrayList$<T>$, crearla vacía es constante. Por consiguiente, la complejidad del total del constructor es O($1$). La otra operación de la clase \textsl{Ciudad} es \textsl{agregarVecino}, que toma un \textsl{Tramo} y lo inserta en la lista de vecinos. Por lo antedicho respecto de la implementación de la lista, agregar un elemento cuesta O($1$).\\
\indent \textbf{Mapa.} Un objeto de esta clase tiene definida las ciudades de origen y destino, así como también un diccionario de todas las ciudades incluídas en el mapa implementado sobre la clase TreeMap$<K,V>$ de Java (donde $K$ es el string nombre de la ciudad, y $V$ es el objeto \textsl{Ciudad}. Mientras que tener individualizados los extremos del recorrido que se busca es fundamental para el algoritmo principal, el diccionario existe sólo por necesidad a la hora de construir eficientemente las distintas ciudades y sus aristas, pues una vez que han sido creadas y conectadas, el algoritmo navega desde los vecinos de origen en adelante, y nunca utiliza este diccionario.\\
\indent La construcción de un \textsl{Mapa} demanda únicamente los strings 'nombre' de la ciudad origen y destino del problema. Primero los utiliza para crear las respectivas ciudades y las asigna a cada campo, lo cual según lo visto hasta aquí es constante. Luego crea un nuevo diccionario vacío, lo cual también demanda tiempo constante. Acto seguido inserta las ciudades origen y destino, lo cual también es O($1$) ya que la cantidad de elementos es constante. Finalmente asigna el diccionario al campo 'ciudades', con lo cual el constructor demanda en su totalidad O($1$).\\
\indent Por otro lado tenemos el método \textsl{agregar}, que toma dos nombres de ciudades y el peso que se puede trasladar entre ellas y crea los objetos necesarios para cargar esa información en el \textsl{Mapa}. Esta función tiene en cuenta distintas variantes: puede que las ciudades ya existan como objeto, con lo cual sólo es necesario crear las aristas correspondientes a la ida y la vuelta entre ambas y agregarlas a las respectivas listas de vecinos o, en su defecto, crear las ciudades y luego sí proceder a crear las aristas y agregarlas. Antes de entrar a la división de casos, la función chequea si las ciudades están definidas en el diccionario. En caso afirmativo devuelve el objeto asociado a la clave, por el contrario devolverá 'null'. Buscar cada ciudad tiene un costo, según la implementación del \textsl{TreeMap}, de O($log n$), donde $n$ indica la cantidad de claves definidas en el diccionario. Dado que las claves son las ciudades, y se definen una única vez, estas operaciones tienen un costo de O($log$ $ciudades$). Luego, los 4 casos que se contemplan son disjuntos, siendo el ùltimo el peor debido a que las ciudades no están definidas y se deben crear los objetos e insertarlos. Se vio que los constructores de \textsl{Ciudad} y \textsl{Arista} son constantes, con lo cual las únicas operaciones que realmente pesan son las que definen las ciudades en el diccionario. Nuevamente, dada la implementación del tipo, esto demanda O($log$ $ciudades$), con lo cual la complejidad de toda la función es O ( $ 4 * log$ $ciudades$) $\in$ O($log$ $ciudades$).

\subsection{Análisis del tiempo de ejecución}












